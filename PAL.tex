\documentclass[a4paper,10pt]{report}

\setcounter{tocdepth}{2}
\fboxsep=0.2in

\begin{document}
\author{Philip J. Roberts and Paul A. Hoadley}
\title{The PAL Abstract Machine \\ An implementation in Java}
\date{}
\maketitle

\tableofcontents
\listoffigures
\listoftables

\chapter{Introduction}
The PAL Abstract Machine is a virtual, stack-based, Harvard
architecture machine.  Data in memory is tagged, so that a type system
is supported explicitly.  The instruction set is short, though
reasonably powerful, including an ``operation'' instruction
(\texttt{OPR}---see Section \ref{sec:instr:OPR}) with some 32
variants, including arithmetic and logical operations, type
conversions and stack manipulation.

This document describes the PAL Abstract Machine in general, and also
an implementation of the machine as a simulator written in the Java
programming language.  The Java simulator supports a human-readable
object file format (described in Section \ref{sec:objectfile}), and
console-based input and output.

\chapter{Architecture}
\section{Memory}
\label{sec:arch.mem}
Storage is divided into two regions: a linear instruction store, and a
data stack.  The instruction store is essentially write-once (at
object file load time) and then read-only: self-modifying code is not
possible.  The first location in the instruction store is at address
1.

The data stack is manipulated by both the user program and the machine
itself, as described in the various instructions listed in Section
\ref{sec:mnemonics}.  In addition to the conventional stack
operations, the data stack provides a degree of random access.
Variables are implemented by referring to a stack location relative to
the current stack frame.  The relation is described in terms of a
\emph{level difference} and a \emph{displacement}.  The level
difference refers to the number of stack frames the target stack frame
lies below the current stack frame.  Clearly, this number can be zero,
in which case the current stack frame itself is referenced.  The
displacement is the number of stack positions that the target position
lies above the top of its stack mark.  The first such position has a
displacement of zero.

Throughout this document, reference is made to a top-of-stack pointer
which is maintained by the machine as a reference to the top item in
the data stack.  This pointer is largely internal to the machine:
although it can be manipulated directly (by the \texttt{INC}
instruction---see Section \ref{sec:instr:INC}), and is moved about
implicitly by other instructions, there is no straightforward way, for
example, to obtain the absolute value of this pointer from within a
program.

Data items of all sizes occupy only one position on the data stack.
For example, an integer and the string \texttt{'an integer'} can both
be stored in a single memory location.  Similarly, an instruction and
all of its operands are stored in a single location in the instruction
store.

The \emph{stack mark} is the area at the bottom of a stack frame
containing meta-information for that stack frame.  Figure
\ref{fig:stack} shows the format of the stack frame used by the PAL
Abstract Machine.  The machine sets up the stack mark for the first
stack frame automatically.

\begin{figure}[htbp]
  \centering
  \input{stackframe.tex}
  \centerline{\box\graph}
  \caption[Stack frame]{Stack frame: the stack mark is shaded}
  \label{fig:stack}
\end{figure}

\section{Type system}
Each item in the data stack is \emph{tagged} with a type.  The type of
a data item affects how that item can be manipulated, as described in
the various instructions in Section \ref{sec:mnemonics}.  The types
are:
\begin{itemize}
\item bool
\item real
\item int
\item string
\item undef
\end{itemize}
Naturally, a value of type bool can be only ``true'' or ``false''.
The value of an item tagged with type undef is undefined.  Values
tagged with types int, real and string are constrained by the same
rules as Java's int, float and string literals respectively.

\section{Exception system}
\label{sec:exceptions}
The PAL Abstract Machine supports a fairly primitive exception
mechanism.  There are three instructions related to exception
generation and handling: \texttt{SIG} (Section \ref{sec:instr:SIG}),
\texttt{REH} (Section \ref{sec:instr:REH}) and \texttt{OPR 0 31}
(Section \ref{sec:instr:OPR31}).  In addition, the two input
instructions \texttt{RDI} (Section \ref{sec:instr:RDI}) and
\texttt{RDR} (Section \ref{sec:instr:RDR}) may raise exceptions when
they encounter unexpected input.

The \texttt{SIG} instruction is used primarily for raising custom
user-defined exceptions.  For example, to raise exception 5, one would
use the command \texttt{SIG 0 5}.  \texttt{SIG} is also used in
exception handlers as we will see later.

The \texttt{REH} instruction tells the machine where to find an
exception handler, and associates that handler with the currently
active stack frame.

\texttt{OPR 0 31} compares the presently active exception to the
integer value on top of the stack.  It pushes a boolean which
indicates whether they match or not.

The following exceptions are defined:
\begin{description}
\item[0 Re-raise the present active signal] This will cause the
  current signal to be re-raised.  Typically used by an exception
  handler as a last resort to deal with an unrecognised exception.
  Hence, the current frame is ignored in the search for a handler when
  \texttt{SIG 0 0} has been used.
\item[1 Program abort] This signal causes instant termination of the
  program.  No exception handler will be called.  (Raising this
  exception differs from terminating a program with \texttt{JMP 0 0}
  in that a non-zero exit status will be returned to the operating
  system, signaling abnormal termination of the simulator.)
\item[2 No return in function] Indicates that no return statement was
  executed by a function.
\item[3 Type mis-match in input] \texttt{RDI} found a non-integer
  value, or \texttt{RDR} found a non-real value.
\item[4 Attempt to read past end of file] The input file was at
  end-of-file before an \texttt{RDI} or \texttt{RDR} instruction.
\end{description}

An exception handler is typically written as follows:
\begin{enumerate}
\item Use \texttt{OPR 0 31} to test whether the exception is one of a
  number of known exceptions.
\item If it is, execute the appropriate handler code.
\item If the exception is unknown, use \texttt{SIG 0 0} to re-raise
  the exception in case there is a handler in a lower stack frame
  which can deal with it.
\end{enumerate}
For example:
\begin{verbatim}
1:    LCI 0 3        handler code begins here
      OPR 0 31
      JIF 0 10  
      .
      .              code to handle input type mismatch
      .
10:   LCI 0 4
      OPR 0 31
      JIF 0 20
      .
      .              code to handle eof
      .
20:   SIG 0 0        unknown exception.
\end{verbatim}

\chapter{Object File Format}
\label{sec:objectfile}
Object files are plain text files, suitable for editing by hand or
generation by machine, such as the output of a compiler.  The object
file grammar is very simple, and is presented partially in Figure
\ref{fig:objgrammar} in Augmented Backus-Naur Form.\footnote{Crocker
  D, Overell P. (1997) ``Augmented BNF for Syntax Specifications: ABNF
  (RFC 2234)'' The Internet Society.
  \texttt{http://www.ietf.org/rfc/rfc2234.txt}}

\begin{figure}[ht]
  \centering
  \begin{verbatim}
objectfile = 1*(line)
line       = (intline / realline / stringline) EOL
intline    = mnemonic WSP integer WSP integer *1(WSP comment) EOL
realline   = mnemonic WSP integer WSP real *1(WSP comment) EOL
stringline = mnemonic WSP integer WSP string *1(WSP comment) EOL
\end{verbatim}
  \caption{Partial ABNF grammar for an object file}
  \label{fig:objgrammar}
\end{figure}

The undefined terminal symbols are largely self-explanatory.
\texttt{EOL} represents the end-of-line character(s).  \texttt{WSP} is
a non-zero number of whitespace (space or horizontal tab) characters.
A \texttt{mnemonic} is a three-letter mnemonic from the PAL
instruction set (presented in Section \ref{sec:mnemonics}).
\texttt{integer} and \texttt{real} correspond to text that can be
parsed as Java \texttt{int} and \texttt{float} types respectively, and
a \texttt{string} is a string of characters delimited by the single
quote character (\texttt{'}).  An optional comment (consisting of any
characters other than \texttt{EOL}) can appear at the end of any line.

\chapter{Instruction set}
\label{sec:mnemonics}
The PAL Abstract Machine instruction set is presented in summary in
Table \ref{tab:mnemonics}.

\begin{table}[hbt]
  \centering
  \caption{PAL Abstract Machine instruction set}
  \label{tab:mnemonics}

  \vspace{0.25in}

  \framebox{
  \begin{tabular}{lccl}
    MST & $L$ & 0 & Mark the stack \\
    CAL & $M$ & $A$ & Procedure call \\
    INC & 0 & $I$ & Increment top-of-stack pointer by $I$ \\
    JIF & 0 & $A$ & Jump if false to address $A$ \\
    JMP & 0 & $A$ & Jump to address $A$ \\
    LCI & 0 & $I$ & Load integer constant onto stack \\
    LCR & 0 & $R$ & Load real constant onto stack \\
    LCS & 0 & $S$ & Load string literal onto stack \\
    LDA & $L$ & $D$ & Load absolute address of variable onto stack \\
    LDI & 0 & 0 & Load value at address indicated by top-of-stack \\
    LDV & $L$ & $D$ & Load value of a variable onto stack \\
    LDU & 0 & 0 & Load undefined value onto stack \\
    OPR & 0 & $I$ & Execute operation $I$ \\
    RDI & $L$ & $D$ & Read a value into an integer variable \\
    RDR & $L$ & $D$ & Read a value into a real variable \\
    STI & 0 & 0 & Load (top-of-stack $-$ 1) into address at top-of-stack \\
    STO & $L$ & $D$ & Store into a variable \\
    SIG & 0 & $I$ & Raise signal $I$ \\
    REH & 0 & $A$ & Register exception handler at address $A$ \\
    & & & \\
    Legend: & & $A$ & An address in the instruction store \\
    & & $D$ & A displacement in the data store \\
    & & $I$ & An integer number \\
    & & $L$ & A level difference \\
    & & $M$ & The number of parameters \\
    & & $R$ & A real number \\
    & & $S$ & A string \\
  \end{tabular}
  }
\end{table}

\section{\texttt{MST} $L$ 0}
This causes the machine to mark the stack frame.  \texttt{MST} is used
in calling a procedure or function.  A program containing a call
should:
\begin{enumerate}
\item Mark the stack using \texttt{MST}.
\item Optionally push any paramters to the call onto the stack.
\item Call the procedure or function using \texttt{CAL}.
\end{enumerate}
The machine constructs the stack mark by:
\begin{enumerate}
\item Pushing the static link
\item Pushing the dynamic link
\item Pushing space for the return point
\item Pushing space for the address of an exception handler
\end{enumerate}

$L$ is the level difference between the call to the procedure or
function and the declaration of the procedure or function, and is used
to calculate the static link.  For example, in the following:
\begin{verbatim}
    procedure A is
    begin
    end;

    procedure B is
    begin
        A;
    end;
\end{verbatim}
the level difference is 1 between the call to A and the declaration of
A.  However, in the following code:
\begin{verbatim}
    procedure A is
        procedure B is
        begin
        end;
    begin
    B;
    end;
\end{verbatim}
the level difference is 0 as procedure B is declared at the same level
from which it is called.

\section{\texttt{CAL} $M$ $A$}
The base of the current stack frame is moved to point at the current
top-of-stack minus the $M$ parameters already on the stack.  The base
of the new activation record is thus the first location \emph{above}
the stack mark (which should have been constructed by the machine via
the \texttt{MST} instruction immediately prior to the call).  The
return address is stored (by the machine) in the stack mark and the
program counter jumps to address $A$.

\section{\texttt{INC} 0 $I$}
\label{sec:instr:INC}
The top-of-stack pointer is incremented by $I$ positions.  Any stack
positions skipped through the increment are given the type undefined.
This is generally used to allocate space for variables.

\section{\texttt{JIF} 0 $A$}
The value at the top-of-stack must be of type bool, otherwise an error
is signalled and the machine will halt.  If the value is boolean false
\emph{and} the address $A$ is within the range of existing
instructions, the program counter jumps to address $A$.  If the value
is boolean false and the address $A$ is out of range, an error is
signalled and the machine will halt.  If the value is boolean true,
this instruction has no effect.

\section{\texttt{JMP} 0 $A$}
If address $A$ is within the range of existing instructions, the
program counter jumps to address $A$, otherwise an error is signalled
and the machine will halt.  The instruction \texttt{JMP 0 0} is the
correct way to terminate a program.

\section{\texttt{LCI} 0 $I$}
The integer value $I$ is pushed onto the top of the stack and tagged
as type integer.

\section{\texttt{LCR} 0 $R$}
The real value $R$ is pushed onto the top of the stack and tagged as
type real.

\section{\texttt{LCS} 0 $S$}
The string value $S$ is pushed onto the top of the stack and tagged
as type string.

\section{\texttt{LDA} $L$ $D$}
The absolute address of the variable at the stack location with level
difference $L$ and displacement $D$ is pushed onto the top of the
stack.  (See Section \ref{sec:arch.mem} for further discussion of the
level difference and displacement addressing scheme.)

\section{\texttt{LDI} 0 0}
The value of the variable whose address is specified in the top of the
stack is loaded into the top of stack position.  The top-of-stack
pointer remains unchanged.

\section{\texttt{LDV} $L$ $D$}
The value of the variable at the stack location with level difference
$L$ and displacement $D$ is pushed onto the top of the stack.

\section{\texttt{LDU} 0 0}
A value of type undef is pushed onto the stack.  

\section{\texttt{OPR} 0 $I$}
\label{sec:instr:OPR}
A summary of the operations provided by the \texttt{OPR} instruction is given in Table \ref{tab:ops}.

\begin{table}[!hbt]
  \centering
  \caption{Operations provided by the \texttt{OPR} instruction}
  \label{tab:ops}

  \vspace{0.25in}

  \framebox{
    \begin{tabular}{ll}
      $I$ & Operation \\
      \hline
      0 & procedure return \\
      1 & function return \\
      2 & negation \\
      3 & addition \\
      4 & subtraction \\
      5 & multiplication \\
      6 & division \\
      7 & exponentiation \\
      8 & string concatenation \\
      9 & odd \\
      10& equality \\
      11& inequality \\
      12& less-than \\
      13& greater-than-or-equal-to \\
      14& greater-than \\
      15& less-than-or-equal-to \\
      16& not (logical complement) \\
      17& true \\
      18& false \\
      19& end-of-file \\
      20& write \\
      21& newline \\
      22& swap the top two elements of the stack \\
      23& duplicate the element on top of the stack \\
      24& drop the element on top of the stack \\
      25& integer-to-real conversion \\
      26& real-to-integer conversion \\
      27& integer-to-string conversion \\
      28& real-to-string conversion \\
      29& logical and \\
      30& logical or \\
      31& test exception \\
    \end{tabular}
  }
\end{table}

\setcounter{subsubsection}{-1}
\subsection{\texttt{OPR 0 0}: procedure return}
Top of stack is returned to the item at which it pointed prior to the
call.  The program counter is set to the return address which is
stored in the procedure stack frame and the base is set to the value
of base before the procedure call.

\subsection{\texttt{OPR 0 1}: function return}
The value on top of the stack is taken to be the return value of the
function.  The top-of-stack pointer is returned to the value at which
it pointed prior to the function call.  The program counter is set to
the return address stored in the function's stack frame and the base
is set to the value of base before the procedure call.  Finally, the
function result is pushed onto the top of the stack.

\subsection{\texttt{OPR 0 2}: negation}
If the value on top of the stack is of type integer or real it is
negated.  Otherwise, an error is issued and the program terminates.

\subsection{\texttt{OPR 0 3}: addition}
If the top two values on the stack are both of type integer or both of
type real, they are removed from the stack and replaced by their sum.
Otherwise, an error is issued and the program terminates.

\subsection{\texttt{OPR 0 4}: subtraction}
If the top two values on the stack are both of type integer or both of
type real, the top two values are removed from the stack, the top
value is subtracted from the next-to-top value and the result is
pushed onto the stack.  Otherwise, an error is issued and the program
terminates.

\subsection{\texttt{OPR 0 5}: multiplication}
If the top two values on the stack are both of type integer or both of
type real, they are removed from the stack and replaced by their
product.  Otherwise, an error is issued and the program terminates.

\subsection{\texttt{OPR 0 6}: division}
If the top two values on the stack are both of type integer or both of
type real, the top two values are removed from the stack, the
next-to-top value is divided by the top value and the result is pushed
onto the stack.  Otherwise, or if division by zero is attempted, an
error is issued and the program terminates.

\subsection{\texttt{OPR 0 7}: exponentiation}
The value on top of the stack must be of type integer and the value at
next to top may be of type integer or real.  The type of the latter
determines the type of the result.  If these conditions are not
satisfied then an error is issued and the program terminates.
Otherwise the top two values are removed from the stack, the next to
top value is raised to the power of the top value and the result is
pushed onto the stack.

\subsection{\texttt{OPR 0 8}: string concatenation}
If the top two values on the stack are both of type string, these
values are removed from the stack, the top value is appended to the
next-to-top value and the result is pushed onto the stack.  Otherwise,
an error is issued and the program terminates.

\subsection{\texttt{OPR 0 9}: odd}
If the value at the top-of-stack is not of type integer, an error is
issued and the program terminates.  Otherwise, if the value is odd,
the value is replaced by the boolean value true, and if the value is
even, the value is replaced by the boolean value false.

\subsection{\texttt{OPR 0 10}: equality}
If the top two values on the stack are both of type integer or both of
type real, they are compared for equality, and the boolean result is
pushed onto the stack.  Otherwise, an error is issued and the program
terminates.

\subsection{\texttt{OPR 0 11}: inequality}
If the top two values on the stack are both of type integer or both of
type real, they are compared for inequality, and the boolean result is
pushed onto the stack.  Otherwise, an error is issued and the program
terminates.

\subsection{\texttt{OPR 0 12}: less-than}
If the top two values on the stack are not both of type integer or
both of type real, an error is issued and the program terminates.
Otherwise, the top two values are removed from the stack.  Then, if
the next-to-top value is less than the top value the boolean value
true is pushed onto the stack, otherwise the boolean value false is
pushed onto the stack.

\subsection{\texttt{OPR 0 13}: greater-than-or-equal-to}
If the top two values on the stack are not both of type integer or
both of type real, an error is issued and the program terminates.
Otherwise, the top two values are removed from the stack.  Then, if
the next-to-top value is greater than or equal to the top value the
boolean value true is pushed onto the stack, otherwise the boolean
value false is pushed onto the stack.

\subsection{\texttt{OPR 0 14}: greater-than}
If the top two values on the stack are not both of type integer or
both of type real, an error is issued and the program terminates.
Otherwise, the top two values are removed from the stack.  Then, if
the next-to-top value is greater than the top value the boolean value
true is pushed onto the stack, otherwise the boolean value false is
pushed onto the stack.

\subsection{\texttt{OPR 0 15}: less-than-or-equal-to}
If the top two values on the stack are not both of type integer or
both of type real, an error is issued and the program terminates.
Otherwise, the top two values are removed from the stack.  Then, if
the next-to-top value is less than or equal to the top value the
boolean value true is pushed onto the stack, otherwise the boolean
value false is pushed onto the stack.

\subsection{\texttt{OPR 0 16}: logical \emph{not}}
If the value on top of the stack is of type bool, it is replaced by
its logical complement.  Otherwise, an error is issued and the program
terminates.

\subsection{\texttt{OPR 0 17}: true}
The boolean value true is pushed onto the stack.

\subsection{\texttt{OPR 0 18}: false}
The boolean value false is pushed onto the stack.

\subsection{\texttt{OPR 0 19}: eof}
If the end of the input file has been reached, the boolean value true
is pushed onto the stack.  Otherwise, the boolean value false is
pushed onto the stack.

\subsection{\texttt{OPR 0 20}: write}
If the value on top of the stack is of type boolean or undefined, an
error is issued and the program terminates.  Otherwise, the value on
top of the stack is removed and written to the output file.

\subsection{\texttt{OPR 0 21}: newline}
A newline is written to the output file.

\subsection{\texttt{OPR 0 22}: swap the top two elements of the stack}
The top two elements on the stack are swapped.

\subsection{\texttt{OPR 0 23}: duplicate the element on top of the stack}
A copy of the top element of the stack is pushed onto the stack.

\subsection{\texttt{OPR 0 24}: drop the element on top of the stack}
The top element of the stack is removed.

\subsection{\texttt{OPR 0 25}: integer-to-real conversion}
If the value on top of the stack is of type integer, it is replaced by
the real representation of the value.  Otherwise, an error is issued
and the program terminates.

\subsection{\texttt{OPR 0 26}: real-to-integer conversion}
If the value on top of the stack is of type real, it is replaced by
the integer representation of the value.  Otherwise, an error is
issued and the program terminates.

\subsection{\texttt{OPR 0 27}: integer-to-string conversion}
If the value on top of the stack is of type integer, it is replaced by
its string representation.  Otherwise, an error is issued and the
program terminates.

\subsection{\texttt{OPR 0 28}: real-to-string conversion}
If the value on top of the stack is of type real, it is replaced by
its string representation.  Otherwise, an error is issued and the
program terminates.

\subsection{\texttt{OPR 0 29}: logical \emph{and}}
If the top two values on the stack are both of type bool, they are
removed from the stack and replaced with the logical \emph{and} of the
two values.  Otherwise, an error is issued and the program terminates.

\subsection{\texttt{OPR 0 30}: logical \emph{or}}
If the top two values on the stack are both of type bool, they are
removed from the stack and replaced with the logical \emph{or} of the
two values.  Otherwise, an error is issued and the program terminates.

\subsection{\texttt{OPR 0 31}: is(exception)}
\label{sec:instr:OPR31}
If the value on top of the stack is not of type integer, an error is
issued and the program terminates.  Otherwise, the top value is
removed from the stack and compared to the number of the currently
active exception.  If the two numbers are equal, the boolean value
true is pushed onto the stack, otherwise the boolean value false is
pushed onto the stack.

\section{\texttt{RDI} $L$ $D$}
\label{sec:instr:RDI}
The machine reads an integer value from input and stores it in the
location indicated by the level difference $L$ and displacement $D$.
If the input is at end-of-file exception 4 is raised.  If the next
line in the input file is not an integer, exception 3 is raised.

\section{\texttt{RDR} $L$ $D$}
\label{sec:instr:RDR}
The machine reads a real value from input and stores it in the
location indicated by the level difference $L$ and displacement $D$.
If the input is at end-of-file exception 4 is raised.  If the next
line in the input file is not a real, exception 3 is raised.

\section{\texttt{STI} 0 0}
Loads the value in (top-of-stack -- 1) into the variable at the
absolute address specified by the value on top of the stack.  The top
two elements are removed from the stack.

\section{\texttt{STO} $L$ $D$}
Load the value on top of the stack into the stack location specified
by the level difference $L$ and displacement $D$.  The top element of
the stack is removed.

\section{\texttt{SIG} 0 $I$}
\label{sec:instr:SIG}
This instruction causes the entire run-time stack to be searched
looking for an exception handler.  If a handler is found, all
activation records down to the frame containing the handler are
discarded, and control is transferred to the handler.  The exception
system is covered in Section \ref{sec:exceptions}.

\section{\texttt{REH} 0 $A$}
\label{sec:instr:REH}
Registers an exception handler at address $A$.  Address $A$ is stored
in the stack mark as a reference to the exception handling code for
this stack frame.  An address of zero indicates that no exception
handler is registered.  The exception system is covered in Section
\ref{sec:exceptions}.

\chapter{Java simulator}
\section{History}
The implementation of the PAL machine in Java described in this
chapter was written by the authors of this document in an effort to
provide a portable implementation of the machine simulator for the
Compiler Construction course in the Department of Computer Science at
the University of Adelaide.  It was written from scratch using the
existing Ada implementation of the machine as a reference.  The
lineage of the Ada implementation can be traced as follows (according
to comments in the Ada source):
\begin{itemize}
\item Original Pascal implementation by Chris Marlin
\item Translation from Pascal to Ada by Michael Oudshoorn
\item Implementation of indirection and exception handling by Kevin Maciunas
\end{itemize}

\section{Using the Java simulator}
The simulator is invoked from the command line:
\begin{verbatim}
> java -jar PAL.jar objectfile
\end{verbatim}
In this example, the simulator will use \texttt{objectfile} as the
object file, and will perform input and output from the console.  If
no object file is specified, the simulator searches for a file named
\texttt{CODE} in the current directory, and complains if that file is
not found.  Input and output redirection can be used in the standard
way.  For example,
\begin{verbatim}
> java -jar PAL.jar objectfile < input > output
\end{verbatim}
will cause the simulator to draw any input from the file
\texttt{input} and write any output to the file \texttt{output}.

\section{Constraints of the implementation}
The simulator imposes the following arbitrary limitations:
\begin{enumerate}
\item The size of the code store is limited to 1000 instructions (and
  their operands).
\item The size of the data store is limited to 500 items.
\end{enumerate}

\end{document}
